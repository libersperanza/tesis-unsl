\chapter{Conclusiones}

En este trabajo hemos abordado el estudio del modelo de espacios m\'etricos con el fin de aplicarlo a un caso real de estudio. Específicamente hemos usado el enfoque de indexaci\'on basado en pivotes para resolver la b\'usqueda de productos similares a uno dado, en un entorno de comercio electr\'onico. Nos propusimos estudiar el comportamiento de las b\'usquedas por rango y de los $k$ vecinos comparando las t\'ecnicas de selecci\'on de pivotes incremental y aleatoria.\\

Podemos resumir el trabajo realizado en los siguientes puntos:

\begin{enumerate}
\item Se hizo un an\'alisis y particionado por categor\'ias de los productos existentes en el sitio de e-commerce de Mercado Libre.
\item Se realiz\'o una aproximaci\'on experimental para determinar el radio de b\'usqueda m\'as adecuado al caso de estudio.
\item Se construyeron \'indices utilizando las t\'ecnicas de selecci\'on aleatoria e incremental.
\item Se implement\'o el algoritmo de b\'usqueda por rango y  se seleccion\'o como criterio para analizar su desempeño, la cantidad de evaluaciones de la funci\'on de distancia.
\end{enumerate}

Se ha demostrado experimentalmente en el presente trabajo, que a medida que aumenta el n\'umero de pivotes que compone el \'indice utilizado para las b\'usquedas, la cantidad de evaluaciones de la funci\'on de distancia disminuye. Esto es independiente de la t\'ecnica de selecci\'on de pivotes utilizada para la construcci\'on de dicho \'indice y del tamaño de la base de datos analizada.\\

Por otro lado, si analizamos los resultados experimentales de las t\'ecnicas de selecci\'on, no es posible observar una diferencia significativa en el comportamiento de ambas, por lo cual no podemos afirmar que una t\'ecnica es mejor que la otra. Esto se contrapone con las conclusiones expuestas en \cite{BNCsccc01}, donde queda demostrado que la t\'ecnica de selecci\'on incremental es mejor que la t\'ecnica de selecci\'on aleatoria.\\

Finalmente, teniendo en cuentas los mejores resultados obtenidos por cada t\'ecnica de selecci\'on de pivotes en cada una de las categor\'ias, se puede concluir que no han tenido un desempeño adecuado como  para ser aplicadas directamente en un sistema de recomendaciones real. Aun as\'i, creemos que los resultados obtenidos son importantes en el sentido de que nos dan una  base sobre la cual empezar a trabajar sobre un sistema real de recomendaciones.\\

Este trabajo no pretende ser concluyente en cuanto a dar respuestas definitivas sobre el caso estudiado, que de hecho es un problema complejo para abordar. El objetivo fue realizar una primera incursi\'on exploratoria al problema para obtener bases sobre las cuales seguir trabajando.\\

Las conclusiones mencionadas pueden ser explicadas por la llamada maldici\'on de la dimensionalidad. La aproximaci\'on de los histogramas de distancia de los espacios conformados por cada una de las categor\'ias dieron indicios de que las mismas constitu\'ian espacios de alta dimensionalidad, donde est\'a demostrado que los algoritmos basados en pivotes no tienen buen desempeño \cite{BNCsccc01}.\\

Por lo expuesto, varias son las aristas que se abren para el trabajo futuro:

\begin{itemize}
\item En funci\'on de que el espacio m\'etrico con el que se trabaja en este \'ambito es de alta dimensionalidad,  se deber\'ia estudiar el comportamiento de los algoritmos basado en particiones compactas con las distintas t\'ecnicas de selecci\'on de centros. Las variables a estudiar depender\'an de los \'indices elegidos pero b\'asicamente hay dos aspectos a considerar: cu\'al \'indice usar y  qu\'e t\'ecnicas de selecci\'on de centros aplicar.
\item  En este trabajo restringimos nuestra aplicaci\'on a un volumen de datos que permitiera el manejo en memoria principal del \'indice, pero el volumen real de datos que maneja un sistema de comercio electr\'onico hace suponer debar\'an usarse t\'ecnicas de indexaci\'on para memoria secundaria. Este es otro aspecto a estudiar.
\item En este trabajo hemos analizado el desempeño del \'indice usando como funci\'on de costo cantidad de evaluaciones de distancias realizadas, que es lo que normalmente se encuentra en bibliograf\'ia. Pero en un sistema de recomendaci\'on es muy importante que tan adecuada es la respuesta dada por el sistema, es decir, no s\'olo hay que estudiar el tiempo insumido en encontrar la respuesta sino tambi\'en medir la calidad de la respuesta encontrada. Este es otro punto a abordar en estudios futuros.
\end{itemize}