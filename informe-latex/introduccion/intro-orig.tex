\chapter{Introducci\'on}
\label{chap:intro}

Secci\'on Vac\'ia

\section{Planteamiento del problema}

\section{Objetivos}

\section{Como se organiza la tesis}
En el cap\'itulo 2 se describir\'an conceptos b\'asicos relacionados a la calidad de la informaci\'on en la Web y en particular en Wikipedia, as\'i como tambi\'en se presentar\'an diferentes enfoques que se han propuesto en trabajos previos. En la subsecci\'on 2.1 se introduce el concepto de calidad de la informaci\'on de forma general. En la subsecci\'on 2.2 se detallar\'an trabajos relacionados a la calidad de la informaci\'on en la Web. En la subsecci\'on 2.3 se describir\'a lo que es considerado de calidad en Wikipedia.

En el cap\'itulo 3 se presentar\'an enfoques que capturan aspectos de calidad de la informaci\'on en los art\'iculos de Wikipedia, los cuales permiten obtener indicadores de caracter\'isticas de calidad en ellos. En la subsecci\'on 3.1 se formalizar\'a el concepto de m\'etrica y se detallar\'an las diferentes m\'etricas de calidad utilizadas en diferentes trabajos.
M\'as adelante, en la subsecci\'on 3.2, se mostrar\'a c\'omo se llevaron a cabo diversas investigaciones para la identificaci\'on de art\'iculos destacados y por \'ultimo en la subsecci\'on 3.3, se explicar\'a la detecci\'on de fallas en Wikipedia y c\'omo se han abordado los diferentes enfoques para detectarlas. 