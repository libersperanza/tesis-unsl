\chapter{Detalles de implementaci\'on}

\section{Introducci\'on}
En el siguiente cap\'itulo describiremos los detalles del trabajo realizado, 
desde la estructuraci\'on de las fuentes de datos utilizados, hasta la 
implementaci\'on de los algoritmos de b\'usqueda.
Dado que el objetivo principal de \'este trabajo es determinar (a trav\'es de 
la aplicaci\'on a un caso de uso real) que estrategia de selecci\'on de pivotes 
es la mejor, nos restringimos a implementar el m\'etodo b\'asico de b\'usqueda que 
puede utilizarse para realizar b\'usquedas desde cualquier tipo de sistema: sitio 
web, aplicaci\'on de tel\'efono celular, etc.
Para el desarrollo del software seleccionamos el framework Grails (versi\'on 
1.3.7), que contiene el lenguaje Groovy para la codificaci\'on y Java para la 
ejecuci\'on.
Todas las estructuras de datos fueron manejadas en memoria principal, tanto 
para la creaci\'on de \'indices como para las b\'usquedas.


\section{Procesamiento inicial de los datos de entrada}

Cuando hablamos de datos de entrada, nos referimos a los productos ofrecidos en
 el sitio MercadoLibre (a los que tambi\'en llamaremos items). Estos productos 
 est\'an clasificados dentro de un \'arbol de categor\'ias, donde cada 
 çcategor\'ia re\'une productos relacionados. Remitiendonos a los n\'umeros, 
 obtuvimos alrededor de 2 millones de productos, distribuidos en 12.000 
 categor\'ias (hojas).
La informaci\'on relevante a \'este trabajo obtenida de los productos fue la 
siguiente: categor\'ia, identificador del producto, t\'itulo, descripci\'on 
principal y descripci\'on secundaria.
Para optimizar el desarrollo de creaci\'on de \'indices, los datos de entrada 
tuvieron que ser pre-procesados para crear los archivos necesarios para el 
correcto funcionamiento de dicho proceso. De este pre-procesamiento obtuvimos:
\begin{enumerate}[(1)]
\item  Archivo de productos: texto plano conteniendo toda la informaci\'on 
pertinente de los productos, separada por comas.
\item  Archivo de categor\'ias: texto plano conteniendo el nombre de la 
categor\'ia.
\item  Archivo de productos divididos por categor\'ia: archivo de datos 
serializados, conteniendo un mapa cuyas claves son las categor\'ias y cuyo 
valor es la lista de productos correspondiente, conteniendo informaci\'on 
m\'inima para la selecci\'on de pivotes: identificador del producto y 
codificaci\'on del t\'itulo (la codificaci\'on consiste en eliminar caracteres 
especiales, reemplazar vocales acentuadas por no acentuadas, suprimir espacios 
superfluos y pasar a may\'usculas el t\'itulo completo).
\end{enumerate}

\section{Estructuras de datos}
Para el almacenamiento de las categor\'ias se implement\'o un rebalse abierto 
lineal con un factor de carga de 0.4.
Cada elemento del rebalse se compone de una estructura que contiene el nombre 
de la categor\'ia y una lista de firmas. Cada firma representa la distancia de 
edici\'on de cada t\'itulo del \'item a cada t\'itulo del elemento pivote, e 
incluye el identificador de \'item en cuesti\'on.
Por otro lado, para el almacenamiento de los datos completos del producto se 
utiliz\'o un mapa cuya clave es el identificador y el valor asociado es una 
estructura que contiene la totalidad de los datos del producto.
Como \'ultima estructura de datos auxiliar, se eligi\'o un mapa cuya clave es 
la categor\'ia y el valor asociado es la lista de pivotes correspondiente, 
conteniendo solo el identificador del producto y la codificaci\'on del t\'itulo.
<<Agregar gr\'aficos de estructuras>>

\section{Software desarrollado}
El desarrollo realizado consta de dos partes complementarias: la funcionalidad 
de creaci\'on de \'indices y la funcionalidad de b\'usqueda propiamente dicha, 
que utiliza los \'indices creados por la primera funcionalidad.

\subsection{Creaci\'on de \'indices para las b\'usquedas}
La funcionalidad de creaci\'on de \'indices se desarroll\'o en forma gen\'erica
, preparada para recibir los siguientes par\'ametros: tipo de pivotes (random o
 incremental), cantidad de pivotes y conjunto de pivotes (mismo conjunto para 
 todas las categor\'ias o diferentes conjuntos para cada categor\'ia)
La l\'ogica b\'asica del algoritmo de creaci\'on de \'indices carga los 
archivos descritos en el punto (*) en las estructuras de datos correspondientes
, y procede a generar las firmas de los items, utilizando en primera instancia 
la estrategia de selecci\'on de pivotes en conjunci\'on con el resto de los 
par\'ametros (cantidad y conjunto de pivotes). Luego de completar la estructura
 con las firmas de los items, almacena cada estructura en un archivo de datos 
 serializados; para que puedan ser utilizados en futuras b\'usquedas.
De \'esta manera, el algoritmo de creaci\'on de indices particiona el universo 
de datos U (los productos) a trav\'es de las categor\'ias hojas, resultando en 
multiples U-i listas que potencialmente contienen elementos relevantes para una
 consulta.

Pseudo-código del proceso de creación de índice 
\begin{lstlisting}
crearIndice(File archivoCategorias, File archivoPivotes, String estrategiaSeleccion, int cantPivotes)
begin
 /*Se inicializan las estructuras*/
 Hash categsHash = inicializarHashCategorias(archivoCategorias)
 Map pivotesPorCateg
 /*Se obtiene el conjunto de pivotes en base a la estrategia*/
 if(estrategiaSeleccion == "random")
 begin
  pivotesPorCateg = obtenerConjuntoPivotesRandom(archivoPivotes, cantPivotes)
 end

 if(estrategiaSeleccion == "incremental")
 begin
  pivotesPorCateg = obtenerConjuntoPivotesIncremental(archivoPivotes, cantPivotes)
 end

 Map items
 /*Se calculan las firmas para todos los items*/
 forall item in archivoItems
 begin
  Pivotes pivs = pivotesPorCateg.get(item.categ)
  Dist[] d = calcularDistancias(item.cod_titulo, pivotes)
  categsHash.get(item.categ).add(d)
  items.put(item.id, item)
 end
 /*Se almacenan las estructuras que componen el \'indice en archivos para su utilizaci\'on futura*/
 generarArchivoSerializado(categsHash)
 generarArchivoSerializado(pivotesPorCateg)
 generarArchivoSerializado(items)
end
\end{lstlisting}

<<Pseudo-c\'odigo de selecci\'on random solo para conjunto distinto de pivotes para cada categor\'ia>>
\begin{lstlisting}
obtenerConjuntoPivotesRandom(File archivoPivotes, int cantPivotes)
begin
 Map conjuntoFinal
 Map pivotesPorCateg = obtenerPivotesPorCateg(archivoPivotes)
 for(categ, pivotes in pivotesPorCateg)
 begin
  /*Se eliminan elementos random hasta alcanzar la cantidad de pivotes deseada*/
  while(pivotes.size > cantPivotes)
  begin
   eliminarElementoRandom(pivotes)
  end
  conjuntoFinal.put(categ, pivotes)
 end
 return conjuntoFinal
end
\end{lstlisting}
<<Pseudo-c\'odigo de selecci\'on incremental solo para conjunto distinto de pivotes para cada categor\'ia>>
\begin{lstlisting}
obtenerConjuntoPivotesIncremental(File archivoPivotes, int cantPivotes)
begin
 Map conjuntoFinal
 Map pivotesPorCateg = obtenerPivotesPorCateg(archivoPivotes)
 for(categ, pivotes in pivotesPorCateg)
 begin
  conjuntoPivotes
  int candidato, selec
  float dim, max
  for(i = 1 to cantPivotes)
  begin
   /* Se elige un candidato inicial */ 
   selec = random(n)
   while(selec ya haya sido elegido antes)
    selec = random(n)
   conjuntoPivotes->puntos[i] = pivotes->puntos[selec]
   /*10 es la cantidad de pares de elementos para el calculo de la media D*/
   max = mediaD(conjuntoPivotes, 10)
   candidato = selec
   for j = 2 to N
    begin
    /* Se busca nuevo cantidato */
    selec = random(n)
    while(selec ya haya sido escogido antes)
     selec=random(n)
    conjuntoPivotes->puntos[i] = pivotes->puntos[selec]
    /*10 es la cantidad de pares de elementos para el calculo de la media D*/
    dim = mediaD(conjuntoPivotes, 10)
    if(dim > max)
    begin
     max = dim
     candidato = selec
    end
   end
   conjuntoPivotes->puntos[i] = pivotes->puntos[candidato]
  end
  conjuntoFinal.put(categ, pivotes)
 end
 return conjuntoFinal
end
\end{lstlisting}

\subsection{Busqueda por similitud}
Se implementaron dos funcionalidades de b\'usqueda: b\'usqueda por rango, que 
recibe por par\'ametro el radio de b\'usqueda, la categor\'ia del producto y el
 t\'itulo de un producto existente; y b\'usqueda de los k-vecinos, que recibe 
 por par\'ametro la cantidad de elementos a retornar (k), la categor\'ia del 
 producto y el t\'itulo de un producto existente.
La b\'usqueda por rango calcula la distancia del t\'itulo del producto dado por
 par\'ametro a cada uno de los pivotes, y luego compara esa firma contra las 
firmas de los productos de la categor\'ia seleccionada. Este paso devuelve los 
productos candidatos a formar parte de la respuesta final, los cuales son 
utilizados para calcular la distancia real de edici\'on contra el producto 
buscado, descartando aquellos que est\'en fuera del rango especificado.
<<pseudo-codigo del proceso de busqueda por rango>>
\begin{lstlisting}
busquedaPorRango(String q, String categ, int radio)
begin
 Pivotes pivotes = pivotesPorCateg.get(categ)
 /*Se calcula la firma para la query*/
 Dist[] d = calcularDistancia(q, pivotes)
 List candidatos = []
 /*Se obtienen todas las firmas para la categoria*/
 List[Dist[]] firmas = categsHash.get(categ)
 /*Se compara la firma de la query con las firmas de la categoria, si el valor es mayor que el radio, se descarta el item*/
 for(j = 0 to firmas.size)
 begin
  int[] dists = firmas[j].dists
  boolean agregar = true
  for (i = 0 to dists.size)
  begin
   int value = valorAbsoluto(sig.dists[i] - dists[i])
   if (value > radio)
   begin
    i = dists.size
    agregar = false
   end
  end
  if(agregar)
   candidatos.add(firmas[j])
 end
 List itemsEncontrados = []
 for(i = 0 to candidatos.size)
 begin
  Item item = items.get(candidatos[i].id)
  int dist = distanciaEdicion(q, item.cod_titulo)
  if((radio - dist) > 0)
   itemsEncontrados.add(item)
 end
 return itemsEncontrados
end
\end{lstlisting}

La b\'usqueda de los k-vecinos utiliza una variaci\'on de la b\'usqueda por 
rango, comenzando con un rango de 5 e incrementando ese valor hasta llegar a la
 cantidad deseada de resultados.
<<pseudo-codigo del proceso de busqueda de los k-vecinos>>
\begin{lstlisting}
busquedaKVecinos(String q, String categ, int radio, int k)
begin
 int radio = 0
 int i = 1
 List resultadoFinal
 List itemsTemp = []
 while(itemsTemp.size < k)
 begin
  radio = potencia(5,i)
  itemsTemp = busquedaPorRango(q, categ, radio)
  /*Si se obtuvieron m\'as elementos que el k deseado, se procede a realizar una bisecci\'on del radio*/
  if(itemsTemp.size > k)
  begin
   int li = potencia(5,i - 1)
   int ls = radio
   /* Se realiza una bisecci\'on del radio*/
   while(li <= ls)
   begin
    radio = ((ls + li)/2)
    itemsTemp = busquedaPorRango(q, categ, radio)
    /*Se alcanz\'o el numero deseado de elementos, se finaliza la bisecci\'on*/
    if(itemsTemp.size == k)
    begin
     li = ls + 1
     resultadoFinal = itemsTemp
    end
    /*Se contin\'ua la bisecci\'on para uno u otro lado, dependiendo de si se super\'o o no la cantidad de elementos*/
    else
    begin
     if(itemsTemp.size < k)
     begin
      li = radio + 1
      radio = radio + 1
     end
     else
      ls = radio - 1
    end
   end
   if(itemsTemp.size != k)
   begin
    /*Si el \'ultimo resultado de la bisecci\'on obtuvo menos elementos, se debe hacer la b\'usqueda con el valor final del radio*/
    if(itemsTemp.size < k)
     itemsTemp = busquedaPorRango(q, categ, radio)
    /*Se ordenan los elementos por su distancia y se seleccionan los k primeros*/
    ordenarPorMenorDistancia(itemsTemp)
    resultadoFinal = itemsTemp.subList(0,k)
   end
  end
  else
   resultadoFinal = itemsTemp
  i = i + 1 
 end
 return resultadoFinal
end
\end{lstlisting}

Como funcionalidad auxiliar, se implement\'o un proceso de carga que puede ser 
utilizado al iniciar el programa, para cargar los archivos de \'indices 
previamente generados.

\section{Re-particionado del universo}
Luego de planificar y diseñar e implementar todas las estructuras de datos, 
realizamos pruebas manuales para asegurarnos del correcto funcionamiento del 
software completo.
Ante \'estas pruebas, detectamos que el particionado no era sem\'anticamente 
correcto, ya que al elegir la categor\'ia hoja del \'arbol, estabamos 
restringiendo demasiado el universo de b\'usqueda. Representando la situaci\'on
 con un ejemplo, supongamos que deseamos recomendar productos similares al 
 producto \textit{“Samsung Galaxy A30 32 GB Blanco 3 GB RAM”}, la categor\'ia 
 hoja de dicho producto es \textit{“A30”}, dentro del \'arbol: 
 \textit{“Celulares y Tel\'efonos > Celulares y Smartphones > Samsung > A30”}, 
 si solo tenemos en cuenta los productos que pertenecen a la categor\'ia 
 \textit{“A30”}, es probable que no encontremos, por ejemplo, celulares blancos
  de 32 GB de la marca Motorola.
Por \'este motivo, definimos volver a particionar el universo de productos, 
\'esta vez utilizando la categor\'ia inicial del \'arbol (Celulares y 
Tel\'efonos en el ejemplo anterior).
Con \'esta nueva estrategia, obtuvimos 30 particiones distintas de nuestro 
universo, para las cuales realizamos los experimentos descritos en el pr\'oximo
 cap\'itulo.
