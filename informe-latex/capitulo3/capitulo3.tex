\chapter{Una Aplicaci\'on de Espacios M\'etricos a Comercio Electr\'onico}

\section{Introducci\'on}

El e-commerce (o comercio electr\'onico) es un modelo de negocio que gestiona compra y venta de productos y/o servicios  y utiliza Internet como principal medio de intercambio. Esto tambi\'en incluye gesti\'on de cobros, pagos y distribuci\'on de productos. El desarrollo tecnol\'ogico ha permitido que las comunicaciones crezcan de una manera exponencial, impactando directamente en el crecimiento del comercio electr\'onico.\\

Dentro de los desaf\'ios que implica construir un sitio de e-commerce exitoso, sin duda uno de los m\'as importantes es contar con un cat\'alogo de productos bien organizado, donde el cliente no solo encuentre r\'apidamente lo que busca, sino que los resultados ofrecidos sean relevantes a lo que necesita.\\

En este trabajo nos focalizamos en dicho problema, utilizando la b\'usqueda por similitud sobre los t\'itulos de los productos, con el fin de obtener productos relacionados.

\section{Problema Abordado: Sistema de Recomendaci\'on}

Con el crecimiento de la web y la democratizaci\'on del negocio, el volumen de datos e informaci\'on que se maneja ha crecido exponencialmente, con lo cual las b\'usquedas se enfrentan a desaf\'ios de rendimiento y velocidad de respuesta, dejando en segundo plano la congruencia de datos que ofrecen los modelos relacionales (b\'usquedas exactas).\\

Actualmente, la forma m\'as conocida y popular de almacenar informaci\'on se basa en las relaciones entre los distintos dominios y su representaci\'on computacional (bases de datos relacionales). En \'este tipo de universo, las consultas vienen dadas por una especificaci\'on de criterios que deben cumplirse para que un objeto forme parte del resultado; pero ¿qu\'e pasa cuando en la consulta no podemos definir ese conjunto de criterios en forma determin\'istica? Un ejemplo de esto podr\'ia ser la consulta: ¿qu\'e objetos dentro de mi universo son similares a un objeto dado?.\\

En el campo del e-commerce, la recomendaci\'on de productos a clientes puede clasificarse dentro de la consulta planteada en el p\'arrafo anterior.\\

Un sistema de recomendaci\'on es un software que filtra informaci\'on de inter\'es para el usuario con el fin de proponerle aquel producto m\'as adecuado a sus necesidades. Estos sistemas eval\'uan cu\'al es el grado de inter\'es de un usuario por ciertos productos y buscan productos similares y con una alta probabilidad de atraer su atenci\'on.\\

Las recomendaciones pretenden ser una forma de ayudar al usuario a encontrar productos de su agrado realizando una preselecci\'on bas\'andose, por ejemplo, en su historial de b\'usquedas. Esto proporciona un alivio a los consumidores, ya que les evita tener que recorrer una lista interminable de ofertas poco relevantes. Por otro lado, se espera que estos beneficios para los usuarios se terminen reflejando en aumentos de tr\'afico y ventas, ya que en el e-commerce, las buenas recomendaciones siempre conducen a incrementos en las compras y, por ende, en los m\'argenes de ganancia. Es por ello que la efectividad del sistema de recomendaci\'on elegido es crucial.\\

El funcionamiento de un sistema de recomendaci\'on est\'a basado en cierta informaci\'on que es procesada por los algoritmos de filtrado; y dependiendo de la naturaleza de dicha informaci\'on, podemos clasificar estos algoritmos en: basados en contenido, colaborativos, sensibles al contexto, entre otros.
Los algoritmos basados en contenido filtran objetos o contenidos similares a los que el usuario ya ha buscado, comprado o calificado positivamente. Por ejemplo, en las plataformas de reproducci\'on de m\'usica online, el software eval\'ua las piezas musicales analizando su estructura interna para encontrar piezas similares que podr\'ian tener una l\'inea de bajo parecida.\

Los algoritmos colaborativos, en cambio, filtran objetos basados en usuarios que han valorado estos objetos de manera similar; es decir, si un usuario muestra inter\'es por determinado producto, el sistema lo recomienda a otros usuarios. Por ello, en este tipo de sistemas no es necesaria la informaci\'on del producto en s\'i. Un ejemplo de la utilizaci\'on de este sistema es Amazon.\\

\section{Problema abordado}

En la plataforma de e-commerce Mercado Libre los usuarios publican productos dentro de un \'arbol de categor\'ias; dichos productos constan de un t\'itulo principal, un t\'itulo secundario y una descripci\'on.
El problema que abordaremos en este trabajo es c\'omo encontrar los productos similares a un producto espec\'ifico, obteniendo de esa manera recomendaciones para los usuarios. Aplicaremos la teor\'ia de espacios m\'etricos a este caso de e-commerce real; donde nuestro universo de datos ser\'a un conjunto de productos disponibles en la plataforma Mercado Libre, y la funci\'on que provee la medida de distancia, ser\'a la distancia de edici\'on (o Levenstein) aplicada al t\'itulo de dichos productos.
El objetivo principal ser\'a encontrar la forma m\'as eficiente de resolver la b\'usqueda por similitud, comparando distintos algoritmos y usando como criterio de eficiencia la cantidad de comparaciones realizadas para obtener los resultados de la b\'usqueda.\\

A la hora de hacer una b\'usqueda podemos tomar dos criterios: buscar en todo el universo, lo cual implica un costo muy alto, o segmentar el conjunto de datos y realizar una b\'usqueda m\'as acotada. En este trabajo nos enfocaremos en este \'ultimo punto.\

Dividiremos la b\'usqueda en dos etapas: filtrar por categor\'ia y realizar la b\'usqueda por similitud en la categor\'ia elegida.
